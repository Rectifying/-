\documentclass[a4paper, 12pt]{article}
\usepackage[UTF8]{ctex}
\usepackage{graphicx}
\begin{document}
  \title{实验报告1}
  \author{\\姓名:郭千纯\\
		\\学号:23020007033\\
		\\课程:系统开发工具基础\\}
  \date{\today\\}
  \maketitle
\pagenumbering{roman}
\tableofcontents
\newpage
\pagenumbering{arabic}
\section{练习内容}
版本控制(Git)与Latex 文档编辑

\section{实例及结果}

\subsection{实例1}
git init:创建一个新的 git 仓库,其数据会存放在一个名为 .git 的目录下

\begin{figure}[h!]
  \centering
  \includegraphics[width=1\textwidth]{实例1.png}
  \caption{实例1}
\end{figure}

\subsection{实例2}
git help <command>: 获取 git 命令的帮助信息

\begin{figure}[h!]
  \centering
  \includegraphics[width=1\textwidth]{实例2.png}
  \caption{实例2}
\end{figure}

\subsection{实例3}
git status: 显示当前的仓库状态

\begin{figure}[h!]
  \centering
  \includegraphics[width=1\textwidth]{实例3.png}
  \caption{实例3}
\end{figure}

\subsection{实例4}
git add <filename>: 添加文件到暂存区

\begin{figure}[h!]
  \centering
  \includegraphics[width=1\textwidth]{实例4.png}
  \caption{实例4}
\end{figure}

\subsection{实例5}
git commit: 创建一个新的提交

\begin{figure}[h!]
  \centering
  \includegraphics[width=1\textwidth]{实例5.png}
  \caption{实例5}
\end{figure}

\subsection{实例6}
git log: 显示历史日志

\begin{figure}[h!]
  \centering
  \includegraphics[width=1\textwidth]{实例6.png}
  \caption{实例6}
\end{figure}

\subsection{实例7}
git log --all --graph --decorate: 可视化历史记录(有向无环图)

\begin{figure}[h!]
  \centering
  \includegraphics[width=1\textwidth]{实例7.png}
  \caption{实例7}
\end{figure}

\subsection{实例8}
git checkout <revision>: 更新 HEAD 和目前的分支

\begin{figure}[h!]
  \centering
  \includegraphics[width=1\textwidth]{实例8.png}
  \caption{实例8}
\end{figure}

\subsection{实例9}
git diff <filename>: 显示与暂存区文件的差异

\begin{figure}[h!]
  \centering
  \includegraphics[width=1\textwidth]{实例9.png}
  \caption{实例9}
\end{figure}

\subsection{实例10}
git diff <revision> <filename>: 显示某个文件两个版本之间的差异

\begin{figure}[h!]
  \centering
  \includegraphics[width=1\textwidth]{实例10.png}
  \caption{实例10}
\end{figure}

\subsection{实例11}
\textbackslash documentclass 命令必须出现在每个 LaTeX 文档的开头。花括号内的文本指定了文档的类型。article 文档类型适合较短的文章

\begin{figure}[h!]
  \centering
  \includegraphics[width=1\textwidth]{实例11.png}
  \caption{实例11}
\end{figure}

\subsection{实例12}
\textbackslash maketitle 命令可以给文档创建标题。你需要指定文档的标题。如果没有指定日期,就会使用现在的时间,作者是可选的。

\begin{figure}[h!]
  \centering
  \includegraphics[width=1\textwidth]{实例12.png}
  \caption{实例12}
\end{figure}

\subsection{实例13}
下列分节命令适用于 article 类型的文档:

\textbackslash section{...}\\
\textbackslash subsection{...}\\
\textbackslash subsubsection{...}\\
\textbackslash paragraph{...}\\
\textbackslash subparagraph{...}

\begin{figure}[h!]
  \centering
  \includegraphics[width=1\textwidth]{实例13.png}
  \caption{实例13}
\end{figure}

\subsection{实例14}
页码可以使用\textbackslash pagenumbering{...} 在阿拉伯数字和罗马数字见切换

\begin{figure}[h!]
  \centering
  \includegraphics[width=1\textwidth]{实例14.png}
  \caption{实例14}
\end{figure}

\subsection{实例15}
\textbackslash newpage 命令会另起一个页面

\begin{figure}[h!]
  \centering
  \includegraphics[width=1\textwidth]{实例15.png}
  \caption{实例15}
\end{figure}

\subsection{实例16}
使用 CTeX 宏包。只需要在文档的前导命令部分添加:\\
\textbackslash usepackage[UTF8]{ctex}就可以了。\\
在编译文档的时侯使用 xelatex 命令,因为它是支持中文字体的。

\begin{figure}[h!]
  \centering
  \includegraphics[width=1\textwidth]{实例16.png}
  \caption{实例16}
\end{figure}

\subsection{实例17}
注意,反斜杠不能通过反斜杠转义(不然就变成了换行了),使用 \textbackslash textbackslash 命令代替。 

\begin{figure}[h!]
  \centering
  \includegraphics[width=1\textwidth]{实例17.png}
  \caption{实例17}
\end{figure}

\subsection{实例18}
在 LaTeX 文档中插入图表。这里我们需要引入 graphicx 包。图片应当是 PDF,PNG,JPEG 或者 GIF 文件。

\begin{figure}[h!]
  \centering
  \includegraphics[width=1\textwidth]{实例18.png}
  \caption{实例18}
\end{figure}

\subsection{实例19}
\textbackslash centering 将图片放置在页面的中央。

\begin{figure}[h!]
  \centering
  \includegraphics[width=1\textwidth]{实例19.png}
  \caption{实例19}
\end{figure}

\subsection{实例20}
\textbackslash includegraphics{...} 命令可以自动将图放置到你的文档中,图片文件应当与 TeX 文件放在同一目录下。

\begin{figure}[h!]
  \centering
  \includegraphics[width=1\textwidth]{实例20.png}
  \caption{实例20}
\end{figure}

\section{解题及感悟}
克隆 本课程网站的仓库\\
克隆仓库:\\
git clone https://github.com/missing-semester-cn/missing-semester-cn.github.io.git\\
cd missing-semester-cn.github.io\\
1.将版本历史可视化并进行探索\\
可视化版本历史: 使用 Git 图形界面工具,如 gitk 或 git log:\\
git log --graph --oneline --all\\
或使用 Git 图形界面工具(如 Sourcetree, GitKraken)来更直观地浏览提交历史。\\
2.是谁最后修改了 README.md 文件?(提示:使用 git log 命令并添加合适的参数)\\
查看 README.md 文件的最后修改者:\\
git log -p README.md\\
在输出的日志中,最后一次修改 README.md 文件的提交者是最接近顶部的条目。\\
3.最后一次修改 \_config.yml 文件中 collections: 行时的提交信息是什么?(提示:使用 git blame 和 git show)\\
查看 \_config.yml 文件中 collections: 行最后一次修改时的提交信息:\\
git blame \_config.yml\\
查找 collections: 行对应的提交哈希值,然后使用以下命令查看提交信息:\\
git show <commit-hash>\\
替换 <commit-hash> 为你从 git blame 中得到的实际提交哈希值。\\\\
使用 Git 时的一个常见错误是提交本不应该由 Git 管理的大文件,或是将含有敏感信息的文件提交给 Git 。尝试向仓库中添加一个文件并添加提交信息,然后将其从历史中删除\\
要向仓库添加一个文件并删除它的历史记录,请按照以下步骤操作:
添加文件:\\
   echo "This is a test file." > testfile.txt\\
   git add testfile.txt\\
   git commit -m "Add testfile.txt"\\
删除文件的历史记录:\\
   使用 git filter-branch git filter-repo来从历史记录中删除文件。git filter-repo 是更现代且推荐的方法(需要先安装)。\\
   使用 git filter-repo:\\
   git filter-repo --path testfile.txt --invert-paths\\
   如果你没有安装 git filter-repo,可以使用 git filter-branch(注意这可能会更复杂且不推荐):\\
   git filter-branch --force --index-filter \\
     'git rm --cached --ignore-unmatch testfile.txt' \\
     --prune-empty --tag-name-filter cat -- --all\\
清理和推送更改:\\
   rm -rf .git/refs/original/\\
   git reflog expire --expire=now --all-ref --rewrite\\
   git gc --prune=now --aggressive\\
   git push origin --force --all\\
   git push origin --force --tags\\
这样你可以确保添加的文件从所有历史记录中删除,并将更改推送到远程仓库。\\\\
从 GitHub 上克隆某个仓库,修改一些文件。当您使用 git stash 会发生什么?当您执行 git log --all --oneline 时会显示什么?通过 git stash pop 命令来撤销 git stash 操作,什么时候会用到这一技巧?\\
1.使用 git stash:\\
   当你执行 `git stash` 时,Git 会将你当前工作目录和暂存区的更改保存到一个新的存储栈中,并恢复到最近一次提交的干净状态。这让你可以在不提交更改的情况下切换分支或处理其他任务。\\
2. 执行 git log --all --oneline:\\
   这个命令会显示所有分支的提交历史,按提交的简短哈希值和提交信息展示。你会看到所有提交的记录,包括你当前分支和其他分支上的提交。\\
3. 使用 git stash pop:\\
   当你执行 `git stash pop` 时,Git 会将最近保存的 stash 应用到你的工作目录中,并将其从 stash 栈中删除。这一技巧常用于在处理临时任务或切换分支后恢复未完成的工作。例如,当你需要临时切换到另一分支修复问题时,`git stash` 可以保存当前的工作状态,完成修复后使用 `git stash pop` 恢复你的更改。\\\\
与其他的命令行工具一样,Git 也提供了一个名为 ~/.gitconfig 配置文件 (或 dotfile)。请在 ~/.gitconfig 中创建一个别名,使您在运行 git graph 时,您可以得到 git log --all --graph --decorate --oneline 的输出结果;\\
要在 `~/.gitconfig` 中创建一个别名,使 `git graph` 运行 `git log --all --graph --decorate --oneline`,请按照以下步骤操作:\\
1. 打开 `~/.gitconfig` 文件(如果文件不存在,可以创建它)。\\
2. 在文件中添加以下配置:\\
   graph = log --all --graph --decorate --oneline\\
3. 保存文件并关闭编辑器。\\
现在,运行 `git graph` 会显示 `git log --all --graph --decorate --oneline` 的输出结果。\\\\
您可以通过执行 git config --global core.excludesfile \~/.gitignore\_global 在 \~/.gitignore\_global 中创建全局忽略规则。配置您的全局 gitignore 文件来自动忽略系统或编辑器的临时文件,例如 .DS\_Store;\\
要创建一个全局 `.gitignore` 文件来自动忽略系统或编辑器的临时文件,例如 `.DS\_Store`,请按照以下步骤操作:\\
1. 创建或编辑全局 `.gitignore` 文件:\\
   打开终端,并运行以下命令来创建或编辑 `\~/.gitignore\_global` 文件:\\
   nano ~/.gitignore\_global\\
   或者使用你喜欢的文本编辑器,比如 `vim`、`code`(Visual Studio Code)等。\\
2. 在 `\~/.gitignore\_global` 文件中添加忽略规则:\\
   在文件中添加要忽略的文件或目录的规则。例如:\\
   Thumbs.db\\
   你可以根据需要添加其他规则来忽略更多文件或目录。\\
3. 保存并关闭文件:\\
   如果你使用的是 `nano`,按 `Ctrl + X` 退出,按 `Y` 确认保存更改,然后按 `Enter` 确认文件名。\\
4. 配置 Git 使用这个全局忽略文件:\\
   运行以下命令来告诉 Git 使用这个全局忽略文件:\\
   git config --global core.excludesfile \~/.gitignore\_global\\
现在,Git 将自动忽略你在 `\~/.gitignore\_global` 文件中指定的文件和目录。在任何 Git 仓库中,这些忽略规则都会生效。\\

\end{document}