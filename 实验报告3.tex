\documentclass[a4paper, 12pt]{article}
\usepackage[UTF8]{ctex}
\usepackage{graphicx}
\begin{document}
  \title{实验报告3}
  \author{\\姓名:郭千纯\\
		\\学号:23020007033\\
		\\课程:系统开发工具基础\\}
  \date{\today\\}
  \maketitle
\pagenumbering{roman}
\tableofcontents
\newpage
\pagenumbering{arabic}
\section{github链接}


\section{练习内容}
命令行环境\\
Python 入门基础\\
Python 视觉应用\\

\section{实例及结果}

\subsection{实例1}
使用matplotlib相关知识绘制一个正方形

\begin{figure}[h!]
  \centering
  \includegraphics[width=1\textwidth]{实例1-1.png}
  \caption{实例1}
\end{figure}

\begin{figure}[h!]
  \centering
  \includegraphics[width=1\textwidth]{实例1-2.png}
  \caption{实例1}
\end{figure}

\subsection{实例2}
使用python基础知识做了一个猜数字的小游戏

\begin{figure}[h!]
  \centering
  \includegraphics[width=1\textwidth]{实例2.png}
  \caption{实例2}
\end{figure}

\subsection{实例3}
使用Pillow库来转换图像格式

\begin{figure}[h!]
  \centering
  \includegraphics[width=1\textwidth]{实例3.png}
  \caption{实例3}
\end{figure}

\subsection{实例4}
打开一个名为input\_image.jpg的图像文件,生成其256x256像素的缩略图,并保存为thumbnail\_image.jpg。

\begin{figure}[h!]
  \centering
  \includegraphics[width=1\textwidth]{实例4.png}
  \caption{实例4}
\end{figure}

\subsection{实例5}
将input\_image.jpg图像复制,并将其粘贴到thumbnail\_image.jpg图像上,粘贴位置为(50, 50)。最后,将合成后的图像保存为result\_image.jpg。

\begin{figure}[h!]
  \centering
  \includegraphics[width=1\textwidth]{实例5.png}
  \caption{实例5}
\end{figure}

\subsection{实例6}
打开一个名为input\_image.jpg的图像文件,将其调整为800x600像素,然后旋转90度,并将处理后的图像保存为processed\_image.jpg。

\begin{figure}[h!]
  \centering
  \includegraphics[width=1\textwidth]{实例6.png}
  \caption{实例6}
\end{figure}

\subsection{实例7}
使用matplotlib库绘制了一条直线和几个点,设置了图形标题和坐标轴标签,并显示了图例和图形。

\begin{figure}[h!]
  \centering
  \includegraphics[width=1\textwidth]{实例7-1.png}
  \caption{实例7}
\end{figure}

\begin{figure}[h!]
  \centering
  \includegraphics[width=1\textwidth]{实例7-2.png}
  \caption{实例7}
\end{figure}

\subsection{实例8}
创建了一个100x100像素的黑色图像(所有像素值为0),然后在图像中心绘制了一个白色的方块(像素值为255)。最后,使用matplotlib显示了这个图像。

\begin{figure}[h!]
  \centering
  \includegraphics[width=1\textwidth]{实例8.png}
  \caption{实例8}
\end{figure}

\subsection{实例9}
计算一个包含数字的列表的平均值,并打印出结果。通过使用sum(numbers)获取列表中所有数字的总和,然后用len(numbers)得到列表的长度,最后计算平均值并输出。

\begin{figure}[h!]
  \centering
  \includegraphics[width=1\textwidth]{实例9.png}
  \caption{实例9}
\end{figure}

\subsection{实例10}
定义一个函数is\_prime,用来检查一个数字是否为质数。如果数字大于1且无法被2到该数字平方根之间的任何整数整除,则该数字为质数。最后,代码测试了数字29并打印了是否为质数的结果。

\begin{figure}[h!]
  \centering
  \includegraphics[width=1\textwidth]{实例10.png}
  \caption{实例10}
\end{figure}

\subsection{实例11}
使用切片操作[::-1]来反转字符串text,然后打印出反转后的字符串结果。

\begin{figure}[h!]
  \centering
  \includegraphics[width=1\textwidth]{实例11.png}
  \caption{实例11}
\end{figure}

\subsection{实例12}
使用列表推导式生成了前10个自然数(从1到10)的平方,并将其存储在列表squares中,最后打印出这个列表。

\begin{figure}[h!]
  \centering
  \includegraphics[width=1\textwidth]{实例12.png}
  \caption{实例12}
\end{figure}

\subsection{实例13}
使用生成器表达式计算了列表numbers中所有偶数的和,并将结果打印出来。\\
num for num in numbers if num \% 2 == 0筛选出列表中的偶数,sum()函数计算这些偶数的总和。

\begin{figure}[h!]
  \centering
  \includegraphics[width=1\textwidth]{实例13.png}
  \caption{实例13}
\end{figure}

\subsection{实例14}
使用列表推导式将列表words中的每个字符串转换为大写字母,并将转换后的结果存储在uppercase\_words中,最后打印出这些大写字母字符串

\begin{figure}[h!]
  \centering
  \includegraphics[width=1\textwidth]{实例14.png}
  \caption{实例14}
\end{figure}

\subsection{实例15}
检查了数字number是否为偶数。通过取模运算number \% 2,如果结果为0,则该数字为偶数。最后,代码打印出检查结果。

\begin{figure}[h!]
  \centering
  \includegraphics[width=1\textwidth]{实例15.png}
  \caption{实例15}
\end{figure}

\subsection{实例16}
使用sum()函数计算了列表numbers中所有数字的总和,并将结果打印出来。

\begin{figure}[h!]
  \centering
  \includegraphics[width=1\textwidth]{实例16.png}
  \caption{实例16}
\end{figure}

\subsection{实例17}
关闭删除会话\\
Tmux detach\\
Tmux attach -t 111\\
Tmux kill-session -t 111\\

\begin{figure}[h!]
  \centering
  \includegraphics[width=1\textwidth]{实例17.png}
  \caption{实例17}
\end{figure}

\subsection{实例18}
重命名会话\\
Tmux rename-session –t 111 333\\

\begin{figure}[h!]
  \centering
  \includegraphics[width=1\textwidth]{实例18.png}
  \caption{实例18}
\end{figure}

\subsection{实例19}
切换会话\\
Tmux new -s 111\\
Tmux detach\\
Tmux new -s 222\\
Tmux switch -t 111\\

\begin{figure}[h!]
  \centering
  \includegraphics[width=1\textwidth]{实例19-1.png}
  \caption{实例19}
\end{figure}

\begin{figure}[h!]
  \centering
  \includegraphics[width=1\textwidth]{实例19-2.png}
  \caption{实例19}
\end{figure}

\subsection{实例20}
创建一个新的会话\\
Tmux\\
Tmux new -s 111\\

\begin{figure}[h!]
  \centering
  \includegraphics[width=1\textwidth]{实例20-1.png}
  \caption{实例20}
\end{figure}

\begin{figure}[h!]
  \centering
  \includegraphics[width=1\textwidth]{实例20-2.png}
  \caption{实例20}
\end{figure}

\section{练习感悟}
命令行环境\\
感悟: 在命令行环境下操作是编程的基础技能之一,它让我们能够高效地执行各种任务。掌握基本的命令行操作,如文件导航、创建、删除和编辑文件,不仅提升了操作系统的使用效率,也帮助我们在开发中处理文件和管理项目。在实践中,我学会了如何使用命令行工具来运行Python脚本,这对于自动化任务和调试代码非常重要。\\\\

Python 入门基础\\
感悟: Python作为一门入门友好的编程语言,具有简洁的语法和强大的功能。通过学习Python的基础知识,如数据类型、控制结构、函数和模块,我对编程的核心概念有了更深入的理解。在编写简单的Python程序时,我体会到了编程逻辑的乐趣和解决问题的成就感。这些基础知识为后续的进阶学习打下了坚实的基础。\\\\

Python 视觉应用\\
感悟: Python在视觉应用方面的能力令人惊叹。通过学习如何处理图像和视频,我意识到了Python在数据处理和计算机视觉领域的强大功能。使用库如Pillow和OpenCV来处理图像和应用视觉算法,让我感受到了Python在实际应用中的灵活性和强大。尤其是在实现图像变换、处理和分析时,我体会到了编程如何将抽象的概念转化为可视化的结果。

\end{document}