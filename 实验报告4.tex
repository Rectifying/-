\documentclass[a4paper, 12pt]{article}
\usepackage[UTF8]{ctex}
\usepackage{graphicx}
\begin{document}
  \title{实验报告4}
  \author{\\姓名:郭千纯\\
		\\学号:23020007033\\
		\\课程:系统开发工具基础\\}
  \date{\today\\}
  \maketitle
\pagenumbering{roman}
\tableofcontents
\newpage
\pagenumbering{arabic}
\section{github链接}


\section{练习内容}
调试及性能分析\\
元编程演示实验\\
PyTorch编程\\

\section{实例及结果}

\subsection{实例1}
使用 print() 进行调试,以追踪变量的值和程序执行流程。

\begin{figure}[h!]
  \centering
  \includegraphics[width=1\textwidth]{实例1.png}
  \caption{实例1}
\end{figure}

\subsection{实例2}
用 import logging 和 logging.basicConfig(level=logging.DEBUG) 设置日志记录。logging 模块可以更好地控制输出内容的级别(如 DEBUG, INFO, WARNING),并能方便地将日志信息保存到文件中。

\begin{figure}[h!]
  \centering
  \includegraphics[width=1\textwidth]{实例2.png}
  \caption{实例2}
\end{figure}

\subsection{实例3}
在代码中插入 import pdb; pdb.set\_trace()。逐步执行代码、检查变量值、设置断点等,帮助定位错误。

\begin{figure}[h!]
  \centering
  \includegraphics[width=1\textwidth]{实例3.png}
  \caption{实例3}
\end{figure}

\subsection{实例4}
timeit 可以精确测量小段代码的执行时间,适合比较不同实现的性能。

\begin{figure}[h!]
  \centering
  \includegraphics[width=1\textwidth]{实例4.png}
  \caption{实例4}
\end{figure}

\subsection{实例5}
cProfile 显示函数调用的时间和次数,有助于识别性能瓶颈。

\begin{figure}[h!]
  \centering
  \includegraphics[width=1\textwidth]{实例5.png}
  \caption{实例5}
\end{figure}

\subsection{实例6}
使用 type 函数动态生成类。

\begin{figure}[h!]
  \centering
  \includegraphics[width=1\textwidth]{实例6.png}
  \caption{实例6}
\end{figure}

\subsection{实例7}
定义一个装饰器来修改函数的输入或输出。

\begin{figure}[h!]
  \centering
  \includegraphics[width=1\textwidth]{实例7.png}
  \caption{实例7}
\end{figure}

\subsection{实例8}
定义一个类装饰器以修改类的行为。

\begin{figure}[h!]
  \centering
  \includegraphics[width=1\textwidth]{实例8.png}
  \caption{实例8}
\end{figure}

\subsection{实例9}
创建一个元类并在类定义中使用它。

\begin{figure}[h!]
  \centering
  \includegraphics[width=1\textwidth]{实例9.png}
  \caption{实例9}
\end{figure}

\subsection{实例10}
使用 eval 执行表达式,或使用 exec 执行多行代码。

\begin{figure}[h!]
  \centering
  \includegraphics[width=1\textwidth]{实例10.png}
  \caption{实例10}
\end{figure}

\subsection{实例11}
使用 torch.tensor() 创建张量。

\begin{figure}[h!]
  \centering
  \includegraphics[width=1\textwidth]{实例11.png}
  \caption{实例11}
\end{figure}

\subsection{实例12}
创建张量并进行基本运算。

\begin{figure}[h!]
  \centering
  \includegraphics[width=1\textwidth]{实例12.png}
  \caption{实例12}
\end{figure}

\subsection{实例13}
使用 requires\_grad=True 追踪张量的梯度。

\begin{figure}[h!]
  \centering
  \includegraphics[width=1\textwidth]{实例13.png}
  \caption{实例13}
\end{figure}

\subsection{实例14}
定义一个神经网络类。

\begin{figure}[h!]
  \centering
  \includegraphics[width=1\textwidth]{实例14.png}
  \caption{实例14}
\end{figure}

\subsection{实例15}
将输入传递给模型,获取输出。

\begin{figure}[h!]
  \centering
  \includegraphics[width=1\textwidth]{实例15.png}
  \caption{实例15}
\end{figure}

\subsection{实例16}
使用 setattr 动态设置对象属性。

\begin{figure}[h!]
  \centering
  \includegraphics[width=1\textwidth]{实例16.png}
  \caption{实例16}
\end{figure}

\subsection{实例17}
使用 getattr 动态访问对象属性。

\begin{figure}[h!]
  \centering
  \includegraphics[width=1\textwidth]{实例17.png}
  \caption{实例17}
\end{figure}

\subsection{实例18}
使用 \_\_call\_\_ 使对象可调用。

\begin{figure}[h!]
  \centering
  \includegraphics[width=1\textwidth]{实例18.png}
  \caption{实例18}
\end{figure}

\subsection{实例19}
使用装饰器为函数添加缓存功能。

\begin{figure}[h!]
  \centering
  \includegraphics[width=1\textwidth]{实例19.png}
  \caption{实例19}
\end{figure}


\subsection{实例20}
使用 classmethod 创建工厂方法。

\begin{figure}[h!]
  \centering
  \includegraphics[width=1\textwidth]{实例20.png}
  \caption{实例20}
\end{figure}

\section{练习感悟}
调试和性能分析是编程过程中不可或缺的部分。通过 `print()`、`logging` 和 `pdb` 等工具,我们可以更好地理解代码的执行过程,快速定位问题。使用 `cProfile` 和 `timeit` 进行性能分析,使我们能够识别性能瓶颈,从而优化代码,提升程序效率。元编程为我们提供了强大的工具,允许在运行时动态创建和修改类、函数和属性。通过实例化类、使用装饰器和元类等技术,我们可以实现更灵活和动态的编程风格。这种灵活性可以让我们的代码更加模块化和可重用。通过 PyTorch 的基础学习,我体会到其强大的深度学习功能和友好的接口设计。使用张量运算和自动微分,我们可以轻松构建和训练神经网络,快速实现各种机器学习模型。PyTorch 的动态计算图特性,使得模型的调试和修改变得更加直观。通过实践不同的实例,我对调试、元编程和 PyTorch 的理解加深了。每一个代码示例不仅是对理论知识的应用,更是对实际问题的解决方案。动手实验能帮助我更好地掌握这些技术。编程是一个不断探索和学习的过程。通过这些练习,我意识到还有许多高级特性和技术等待我去发现。保持好奇心和学习热情,将对我的编程能力和项目实践产生积极的影响。总之,本次练习让我在调试、性能分析、元编程和 PyTorch 的应用上获得了新的视角和深刻的理解。未来我会继续深入研究这些领域,并将所学知识应用到实际项目中,以提高自己的编程水平和解决问题的能力。

\end{document}